% Options for packages loaded elsewhere
\PassOptionsToPackage{unicode}{hyperref}
\PassOptionsToPackage{hyphens}{url}
%
\documentclass[
  a4paper]{article}
\usepackage{lmodern}
\usepackage{amssymb,amsmath}
\usepackage{ifxetex,ifluatex}
\ifnum 0\ifxetex 1\fi\ifluatex 1\fi=0 % if pdftex
  \usepackage[T1]{fontenc}
  \usepackage[utf8]{inputenc}
  \usepackage{textcomp} % provide euro and other symbols
\else % if luatex or xetex
  \usepackage{unicode-math}
  \defaultfontfeatures{Scale=MatchLowercase}
  \defaultfontfeatures[\rmfamily]{Ligatures=TeX,Scale=1}
\fi
% Use upquote if available, for straight quotes in verbatim environments
\IfFileExists{upquote.sty}{\usepackage{upquote}}{}
\IfFileExists{microtype.sty}{% use microtype if available
  \usepackage[]{microtype}
  \UseMicrotypeSet[protrusion]{basicmath} % disable protrusion for tt fonts
}{}
\makeatletter
\@ifundefined{KOMAClassName}{% if non-KOMA class
  \IfFileExists{parskip.sty}{%
    \usepackage{parskip}
  }{% else
    \setlength{\parindent}{0pt}
    \setlength{\parskip}{6pt plus 2pt minus 1pt}}
}{% if KOMA class
  \KOMAoptions{parskip=half}}
\makeatother
\usepackage{xcolor}
\IfFileExists{xurl.sty}{\usepackage{xurl}}{} % add URL line breaks if available
\IfFileExists{bookmark.sty}{\usepackage{bookmark}}{\usepackage{hyperref}}
\hypersetup{
  pdftitle={TRAINING Computer exam BFVH15DAVUR},
  pdfauthor={YOUR NAME (YOUR STUDENT NUMBER)},
  hidelinks,
  pdfcreator={LaTeX via pandoc}}
\urlstyle{same} % disable monospaced font for URLs
\usepackage{color}
\usepackage{fancyvrb}
\newcommand{\VerbBar}{|}
\newcommand{\VERB}{\Verb[commandchars=\\\{\}]}
\DefineVerbatimEnvironment{Highlighting}{Verbatim}{commandchars=\\\{\}}
% Add ',fontsize=\small' for more characters per line
\usepackage{framed}
\definecolor{shadecolor}{RGB}{248,248,248}
\newenvironment{Shaded}{\begin{snugshade}}{\end{snugshade}}
\newcommand{\AlertTok}[1]{\textcolor[rgb]{0.94,0.16,0.16}{#1}}
\newcommand{\AnnotationTok}[1]{\textcolor[rgb]{0.56,0.35,0.01}{\textbf{\textit{#1}}}}
\newcommand{\AttributeTok}[1]{\textcolor[rgb]{0.77,0.63,0.00}{#1}}
\newcommand{\BaseNTok}[1]{\textcolor[rgb]{0.00,0.00,0.81}{#1}}
\newcommand{\BuiltInTok}[1]{#1}
\newcommand{\CharTok}[1]{\textcolor[rgb]{0.31,0.60,0.02}{#1}}
\newcommand{\CommentTok}[1]{\textcolor[rgb]{0.56,0.35,0.01}{\textit{#1}}}
\newcommand{\CommentVarTok}[1]{\textcolor[rgb]{0.56,0.35,0.01}{\textbf{\textit{#1}}}}
\newcommand{\ConstantTok}[1]{\textcolor[rgb]{0.00,0.00,0.00}{#1}}
\newcommand{\ControlFlowTok}[1]{\textcolor[rgb]{0.13,0.29,0.53}{\textbf{#1}}}
\newcommand{\DataTypeTok}[1]{\textcolor[rgb]{0.13,0.29,0.53}{#1}}
\newcommand{\DecValTok}[1]{\textcolor[rgb]{0.00,0.00,0.81}{#1}}
\newcommand{\DocumentationTok}[1]{\textcolor[rgb]{0.56,0.35,0.01}{\textbf{\textit{#1}}}}
\newcommand{\ErrorTok}[1]{\textcolor[rgb]{0.64,0.00,0.00}{\textbf{#1}}}
\newcommand{\ExtensionTok}[1]{#1}
\newcommand{\FloatTok}[1]{\textcolor[rgb]{0.00,0.00,0.81}{#1}}
\newcommand{\FunctionTok}[1]{\textcolor[rgb]{0.00,0.00,0.00}{#1}}
\newcommand{\ImportTok}[1]{#1}
\newcommand{\InformationTok}[1]{\textcolor[rgb]{0.56,0.35,0.01}{\textbf{\textit{#1}}}}
\newcommand{\KeywordTok}[1]{\textcolor[rgb]{0.13,0.29,0.53}{\textbf{#1}}}
\newcommand{\NormalTok}[1]{#1}
\newcommand{\OperatorTok}[1]{\textcolor[rgb]{0.81,0.36,0.00}{\textbf{#1}}}
\newcommand{\OtherTok}[1]{\textcolor[rgb]{0.56,0.35,0.01}{#1}}
\newcommand{\PreprocessorTok}[1]{\textcolor[rgb]{0.56,0.35,0.01}{\textit{#1}}}
\newcommand{\RegionMarkerTok}[1]{#1}
\newcommand{\SpecialCharTok}[1]{\textcolor[rgb]{0.00,0.00,0.00}{#1}}
\newcommand{\SpecialStringTok}[1]{\textcolor[rgb]{0.31,0.60,0.02}{#1}}
\newcommand{\StringTok}[1]{\textcolor[rgb]{0.31,0.60,0.02}{#1}}
\newcommand{\VariableTok}[1]{\textcolor[rgb]{0.00,0.00,0.00}{#1}}
\newcommand{\VerbatimStringTok}[1]{\textcolor[rgb]{0.31,0.60,0.02}{#1}}
\newcommand{\WarningTok}[1]{\textcolor[rgb]{0.56,0.35,0.01}{\textbf{\textit{#1}}}}
\usepackage{graphicx,grffile}
\makeatletter
\def\maxwidth{\ifdim\Gin@nat@width>\linewidth\linewidth\else\Gin@nat@width\fi}
\def\maxheight{\ifdim\Gin@nat@height>\textheight\textheight\else\Gin@nat@height\fi}
\makeatother
% Scale images if necessary, so that they will not overflow the page
% margins by default, and it is still possible to overwrite the defaults
% using explicit options in \includegraphics[width, height, ...]{}
\setkeys{Gin}{width=\maxwidth,height=\maxheight,keepaspectratio}
% Set default figure placement to htbp
\makeatletter
\def\fps@figure{htbp}
\makeatother
\setlength{\emergencystretch}{3em} % prevent overfull lines
\providecommand{\tightlist}{%
  \setlength{\itemsep}{0pt}\setlength{\parskip}{0pt}}
\setcounter{secnumdepth}{-\maxdimen} % remove section numbering

\title{TRAINING Computer exam BFVH15DAVUR}
\usepackage{etoolbox}
\makeatletter
\providecommand{\subtitle}[1]{% add subtitle to \maketitle
  \apptocmd{\@title}{\par {\large #1 \par}}{}{}
}
\makeatother
\subtitle{Data Analysis and Visualization using R}
\author{YOUR NAME (YOUR STUDENT NUMBER)}
\date{June 2016}

\begin{document}
\maketitle

\hypertarget{test-header}{%
\subsection{Test header}\label{test-header}}

\begin{itemize}
\tightlist
\item
  \textbf{Teacher} Michiel Noback (NOMI), to be reached at +31 50 595
  4691
\item
  \textbf{Test size} 4 pages; 7 questions
\item
  \textbf{Aiding materials} Computer on the BIN network
\item
  \textbf{Data files}

  \begin{itemize}
  \tightlist
  \item
    \texttt{food\_constituents.txt}
  \end{itemize}
\item
  \textbf{Supplementary materials}

  \begin{itemize}
  \tightlist
  \item
    \texttt{TRAINING\_EXAM.pdf} This test as pdf
  \item
    \texttt{TRAINING\_EXAM.Rmd} This test as R markdown
  \item
    \texttt{R\_cheatsheet.pdf} Lists all R functions that may be used
  \item
    \texttt{rmarkdown-reference.pdf} R markdown reference document
  \end{itemize}
\end{itemize}

\hypertarget{instructions}{%
\subsection{Instructions}\label{instructions}}

In the real test, you should be logged in as guest (username = ``gast'',
password = ``gast''). On your desktop you will find all supplied data
and supplements, as well as the submit script
\texttt{submit\_your\_work}. For this training test, simply quit your
browser and time your work; in the real exam, you will have two hours to
solve a set of similar questions. Use the supplied R markdown file
\texttt{TRAINING\_EXAM.Rmd} to solve and answer the questions of this
test. Fill in your name and student number in the header of this
document. \textbf{Note: never use \texttt{echo\ =\ False} in your code
chunk headers.}\\
All questions have the possible number of points to be scored indicated.
your grade will be calculated as\\
\(Grade = 1 + (\frac{PointsScored}{MaximumScore} * 9)\)

After finishing, \texttt{knit} the result into a pdf document and rename
it to \texttt{TRAINING\_EXAM\_YOUR\_NAME.pdf}.

\hypertarget{data-description}{%
\subsection{Data description}\label{data-description}}

This test explores a dataset containing measurements of several food
constituents in a variety of foods, categorized over several groups.

\hypertarget{code-book}{%
\subsubsection{Code ``Book''}\label{code-book}}

These are the columns, and their descriptions, included in the data file
\texttt{food\_constituents.txt}:\\
id.nr Type kcal protein carb.total carb.sugar carb.other fat.total
fat.sat fat.unsat fiber Na 2 chocolate 442 5.00 67.40 64.60 2.80 15.50
9.00 6.50 6.60 0.100

\begin{enumerate}
\def\labelenumi{\arabic{enumi}.}
\tightlist
\item
  \textbf{id.nr} simple measurement counter
\item
  \textbf{Type} food group
\item
  \textbf{kcal} energy contents in kcal/100g product
\item
  \textbf{protein} protein content in g/100g product
\item
  \textbf{carb.total} total carbohydrate content in g/100g product
\item
  \textbf{carb.sugar} sugar carbohydrates in g/100g product
\item
  \textbf{carb.other} other carbohydrates in g/100g product
\item
  \textbf{fat.total} total fat content in g/100g product
\item
  \textbf{fat.sat} saturated fats in g/100g product
\item
  \textbf{fat.unsat} unsaturated fats in g/100g product
\item
  \textbf{fiber} fiber contents in g/100g product
\item
  \textbf{Na} Sodium content in g/100g product
\end{enumerate}

\hypertarget{here-starts-the-actual-test}{%
\section{Here starts the actual
test}\label{here-starts-the-actual-test}}

\hypertarget{part-1-data-loading-and-cleaning}{%
\subsection{Part 1: Data loading and
cleaning}\label{part-1-data-loading-and-cleaning}}

\hypertarget{question-1-10-points}{%
\paragraph{Question 1 (10 points)}\label{question-1-10-points}}

Load the data from file \texttt{food\_constituents.txt} and assign it to
a variable called \texttt{foods}. Take special care with missing/invalid
fields, and also make sure the columns are loaded in the right data
type.

\begin{Shaded}
\begin{Highlighting}[]
\CommentTok{#your code here}
\end{Highlighting}
\end{Shaded}

If you fail to load the data as instructed above, you may load the
pre-processed file using the following code chunk (uncomment the R
code). Make sure your working directory is set appropriately! You will
not get any points for this question, however.

\begin{Shaded}
\begin{Highlighting}[]
\CommentTok{## Uncomment this line to load pre-processed data}
\CommentTok{#load("./foods_raw.Rdata")}
\end{Highlighting}
\end{Shaded}

\hypertarget{question-2-5-points}{%
\paragraph{Question 2 (5 points)}\label{question-2-5-points}}

There are several rows with missing data. Report these and also remove
these from the \texttt{foods} dastaset. Hint: use the function
\texttt{complete.cases()} to achieve this.

\begin{Shaded}
\begin{Highlighting}[]
\CommentTok{#your code here}
\end{Highlighting}
\end{Shaded}

\hypertarget{part-2-data-exploration}{%
\subsection{Part 2: Data exploration}\label{part-2-data-exploration}}

\hypertarget{question-3-6-points}{%
\paragraph{Question 3 (6 points)}\label{question-3-6-points}}

\textbf{Question 3 a (2 points)} What is the average caloric value of
this food listing?

\begin{Shaded}
\begin{Highlighting}[]
\CommentTok{#your code here}
\end{Highlighting}
\end{Shaded}

\textbf{Question 3 b (2 points)} Tabulate the frequencies of the
different food categories (e.g.~Type)

\begin{Shaded}
\begin{Highlighting}[]
\CommentTok{#your code here}
\end{Highlighting}
\end{Shaded}

\textbf{Question 3 c (2 points)} Show the ``6-number summary'' for
-only- the fat measurements.

\begin{Shaded}
\begin{Highlighting}[]
\CommentTok{#your code here}
\end{Highlighting}
\end{Shaded}

\hypertarget{question-4-12-points}{%
\paragraph{Question 4 (12 points)}\label{question-4-12-points}}

\textbf{Question 4 a (4 points)} Create a new column called
\texttt{fat.cat} that divides the foods into 3 food categories based on
total fat content: \texttt{high.fat}, \texttt{medium.fat} and
\texttt{low.fat}. Take into account that this is an ordinal scale!.

\begin{Shaded}
\begin{Highlighting}[]
\CommentTok{#your code here}
\end{Highlighting}
\end{Shaded}

If you are not able to create this factor, load it from file and attach
it to your foods dataframe. You will not get points for this question of
course.

\begin{Shaded}
\begin{Highlighting}[]
\CommentTok{##uncomment this if you could not create the factor yourself}
\CommentTok{#load("foods_fat_cat.RData")}
\end{Highlighting}
\end{Shaded}

\textbf{Question 4 b (4 points)} Calculate mean energy content for each
fat.cat category.

\begin{Shaded}
\begin{Highlighting}[]
\CommentTok{#your code here}
\end{Highlighting}
\end{Shaded}

\textbf{Question 4 c (8 points) -Challenge question-} Report which foods
from each fat.cat group have the largest fraction of saturated fat
relative to total fat.

\begin{Shaded}
\begin{Highlighting}[]
\CommentTok{#your code here}
\end{Highlighting}
\end{Shaded}

Is there anything funny in these results? Discuss/explain these!

\hypertarget{question-5-8-points}{%
\paragraph{Question 5 (8 points)}\label{question-5-8-points}}

Sort (and list) the Pasta foods by energy content, from high to low.

\begin{Shaded}
\begin{Highlighting}[]
\CommentTok{#your code here}
\end{Highlighting}
\end{Shaded}

\hypertarget{part-3-visualization}{%
\subsection{Part 3: Visualization}\label{part-3-visualization}}

\hypertarget{question-6-8-points}{%
\paragraph{Question 6 (8 points)}\label{question-6-8-points}}

Create a -well annotated- box plot showing distributions of total total
carbohydrate content for the three fat categories (low.fat, medium.fat
and high.fat).

\begin{Shaded}
\begin{Highlighting}[]
\CommentTok{#your code here}
\end{Highlighting}
\end{Shaded}

\hypertarget{question-7-15-points}{%
\paragraph{Question 7 (15 points)}\label{question-7-15-points}}

Create a -well annotated- scatter plot exploring the total carbohydrate
content relative to energy content. You should add a linear regression
line to emphasise the relationship.

\begin{Shaded}
\begin{Highlighting}[]
\CommentTok{#your code here}
\end{Highlighting}
\end{Shaded}

Is there a clear relationship as you would expect? If not, can you
explain?

\end{document}
