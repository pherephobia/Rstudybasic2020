\documentclass[10pt,ignorenonframetext,]{beamer}
\usefonttheme{serif} % use mainfont rather than sansfont for slide text
\setbeamertemplate{caption}[numbered]
\setbeamertemplate{caption label separator}{: }
\setbeamercolor{caption name}{fg=normal text.fg}
\usepackage{lmodern}
\usepackage{amssymb,amsmath}
\usepackage{ifxetex,ifluatex}
\usepackage{fixltx2e} % provides \textsubscript
\ifnum 0\ifxetex 1\fi\ifluatex 1\fi=0 % if pdftex
  \usepackage[T1]{fontenc}
  \usepackage[utf8]{inputenc}
\else % if luatex or xelatex
  \ifxetex
    \usepackage{mathspec}
  \else
    \usepackage{fontspec}
  \fi
  \defaultfontfeatures{Ligatures=TeX,Scale=MatchLowercase}
  \newcommand{\euro}{€}
    \setmainfont[]{NanumSquare}
\fi
% use upquote if available, for straight quotes in verbatim environments
\IfFileExists{upquote.sty}{\usepackage{upquote}}{}
% use microtype if available
\IfFileExists{microtype.sty}{%
\usepackage{microtype}
\UseMicrotypeSet[protrusion]{basicmath} % disable protrusion for tt fonts
}{}
\usepackage{natbib}
\bibliographystyle{apsa-leeper.bst}

% Comment these out if you don't want a slide with just the
% part/section/subsection/subsubsection title:
\AtBeginPart{
  \let\insertpartnumber\relax
  \let\partname\relax
  \frame{\partpage}
}
\AtBeginSection{
  \let\insertsectionnumber\relax
  \let\sectionname\relax
  \frame{\sectionpage}
}
\AtBeginSubsection{
  \let\insertsubsectionnumber\relax
  \let\subsectionname\relax
  \frame{\subsectionpage}
}

\setlength{\emergencystretch}{3em}  % prevent overfull lines
\providecommand{\tightlist}{%
  \setlength{\itemsep}{0pt}\setlength{\parskip}{0pt}}
\setcounter{secnumdepth}{0}

\title{Ch. 1. Statistical Models and Social Science}
\subtitle{Fox., John. 2016. \emph{Applied Regression Analysis and Generalized
Linear Models}, 3rd Ed.}
\author{Sanghoon Park}
\date{2020-09-03}
\usepackage{kotex}
\usepackage[utf8]{inputenc}
\usepackage{tikz}
\usetikzlibrary{through,calc,angles,quotes, arrows.meta,patterns,intersections}

%% Here's everything I added.
%%--------------------------

\usepackage{graphicx}
\usepackage{rotating}
%\setbeamertemplate{caption}[numbered]
\usepackage{hyperref}
\usepackage{caption}
\usepackage[normalem]{ulem}
%\mode<presentation>
\usepackage{wasysym}
%\usepackage{amsmath}


% Get rid of navigation symbols.
%-------------------------------
\setbeamertemplate{navigation symbols}{}

% Optional institute tags and titlegraphic.
% Do feel free to change the titlegraphic if you don't want it as a Markdown field.
%----------------------------------------------------------------------------------
\institute{Department of Political Science\\
University of South Carolina\\}

% \titlegraphic{\includegraphics[width=0.3\paperwidth]{\string~/Dropbox/teaching/clemson-academic.png}} % <-- if you want to know what this looks like without it as a Markdown field. 
% -----------------------------------------------------------------------------------------------------


% Some additional title page adjustments.
%----------------------------------------
\setbeamertemplate{title page}[empty]
%\date{}
\setbeamerfont{subtitle}{size=\small}

\setbeamercovered{transparent}

% Some optional colors. Change or add as you see fit.
%---------------------------------------------------
\definecolor{uofscgarnet}{HTML}{522D80}
 \definecolor{uofscscarlet}{HTML}{F66733}
\definecolor{uiucblue}{HTML}{003C7D}
\definecolor{uiucorange}{HTML}{F47F24}
\definecolor{uofscgarnet}{HTML}{9a341d}
\definecolor{uofscscarlet}{HTML}{cd380e}

% Some optional color adjustments to Beamer. Change as you see fit.
%------------------------------------------------------------------
\setbeamercolor{frametitle}{fg=uofscgarnet,bg=white}
\setbeamercolor{title}{fg=uofscgarnet,bg=white}
\setbeamercolor{local structure}{fg=uofscgarnet}
\setbeamercolor{section in toc}{fg=uofscgarnet,bg=white}
 \setbeamercolor{subsection in toc}{fg=uofscscarlet,bg=white}
\setbeamercolor{footline}{fg=uofscgarnet!50, bg=white}
\setbeamercolor{block title}{fg=uofscscarlet,bg=white}


\let\Tiny=\tiny


% Sections and subsections should not get their own damn slide.
%--------------------------------------------------------------
\AtBeginPart{}
\AtBeginSection{}
\AtBeginSubsection{}
\AtBeginSubsubsection{}

% Suppress some of Markdown's weird default vertical spacing.
%------------------------------------------------------------
\setlength{\emergencystretch}{0em}  % prevent overfull lines
\setlength{\parskip}{0pt}


% Allow for those simple two-tone footlines I like. 
% Edit the colors as you see fit.
%--------------------------------------------------
\defbeamertemplate*{footline}{my footline}{%
    \ifnum\insertpagenumber=1
    \hbox{%
        \begin{beamercolorbox}[wd=\paperwidth,ht=.8ex,dp=1ex,center]{}%
      % empty environment to raise height
        \end{beamercolorbox}%
    }%
    \vskip0pt%
    \else%
        \Tiny{%
            \hfill%
		\vspace*{1pt}%
            \insertframenumber/\inserttotalframenumber \hspace*{0.1cm}%
            \newline%
            \color{uofscgarnet}{\rule{\paperwidth}{0.4mm}}\newline%
            \color{uofscscarlet}{\rule{\paperwidth}{.4mm}}%
        }%
    \fi%
}

% Various cosmetic things, though I must confess I forget what exactly these do and why I included them.
%-------------------------------------------------------------------------------------------------------
\setbeamercolor{structure}{fg=blue}
\setbeamercolor{local structure}{parent=structure}
\setbeamercolor{item projected}{parent=item,use=item,fg=uofscgarnet,bg=white}
\setbeamercolor{enumerate item}{parent=item}

% Adjust some item elements. More cosmetic things.
%-------------------------------------------------
\setbeamertemplate{itemize item}{\color{uofscgarnet}$\bullet$}
\setbeamertemplate{itemize subitem}{\color{uofscgarnet}\scriptsize{$\bullet$}}
\setbeamertemplate{itemize/enumerate body end}{\vspace{.6\baselineskip}} % So I'm less inclined to use \medskip and \bigskip in Markdown.

% Automatically center images
% ---------------------------
% Note: this is for ![](image.png) images
% Use "fig.align = "center" for R chunks

\usepackage{etoolbox}

\AtBeginDocument{%
  \letcs\oig{@orig\string\includegraphics}%
  \renewcommand<>\includegraphics[2][]{%
    \only#3{%
      {\centering\oig[{#1}]{#2}\par}%
    }%
  }%
}

% I think I've moved to xelatex now. Here's some stuff for that.
% --------------------------------------------------------------
% I could customize/generalize this more but the truth is it works for my circumstances.

\ifxetex
\setbeamerfont{title}{family=\fontspec{Titillium Web}}
\setbeamerfont{frametitle}{family=\fontspec{Titillium Web}}
\usepackage[font=small,skip=0pt]{caption}
 \else
 \fi

% Okay, and begin the actual document...

\begin{document}
\frame{\titlepage}

\hypertarget{statistical-models-and-social-reality}{%
\section{Statistical Models and Social
Reality}\label{statistical-models-and-social-reality}}

\begin{frame}{Statistical Models and Social Reality}
\protect\hypertarget{statistical-models-and-social-reality-1}{}

``모든 모델은 틀렸다(wrong), 다만 몇몇 모델만이 유용할 뿐이다'' (Box
1979: 202).

\begin{itemize}
\tightlist
\item
  복잡한 현실을 특정한 모형으로 모두 설명하는 것은 불가능
\item
  사회 구조 혹은 다른 체계적 요인들이 개인의 선택을 조건짓고, 제약
\item
  특정한 개인을 예측하는 것이 아니라 일반적 경향성을 포착하고자 함.
\end{itemize}

사회과학에는 수많은 `추상적인' 이론들이 존재

\begin{itemize}
\tightlist
\item
  추상적인 사회이론들이 만약 복잡한 현실세계를 이해하는 데 유용한가?
\item
  만약 유용하다면, 이와 같은 이론(논리)과 현실(경험)을 어떻게 연결할 수
  있을까?
\end{itemize}

\end{frame}

\begin{frame}{Statistical Models and Social Reality}
\protect\hypertarget{statistical-models-and-social-reality-2}{}

이론은 특정 현상들을 `정의'(define)하고, 그 정의된 현상들 간의 관계를
서술, 설명, 및 예측

\begin{itemize}
\item
  우리는 이 추상적인 이론을 구성하는 각 부분들이 무엇인지 설명하고,
  평가하거나 측정하는 과정을 통해 이론과 현실을 연결(bridging) \pause

  \begin{itemize}
  \tightlist
  \item
    예 ) 교육 수준\(\uparrow\) \(\rightarrow\) 소득 수준\(\uparrow\)
    \pause

    \begin{itemize}
    \tightlist
    \item
      교육 수준과 소득 수준을 어떻게 정의할 것인가?
    \item
      교육 수준과 소득 수준의 관계를 어떻게 설명할 것인가?
    \item
      교육 수준과 소득 수준을 실제로 어떻게 측정할 것인가?
    \item
      이 관계에 영향을 미치는 다른 요인은 존재하는가? 존재한다면
      무엇인가?
    \end{itemize}
  \end{itemize}
\end{itemize}

\end{frame}

\begin{frame}{Statistical Models and Social Reality}
\protect\hypertarget{statistical-models-and-social-reality-3}{}

통계모델: 복잡한 현실을 단순하게(\emph{simplified}) 서술한 것
\citep[2]{Fox2016}

\begin{itemize}
\item
  고용노동자들에 대한 Large-N 표본의 설문조사 자료
\item
  교육 수준, 성별, 인종, 거주지역 등과 같은 특성들로 개인의 소득에 대한
  회귀분석을 시행하였다고 가정
\end{itemize}

\pause 과연 위의 모델로 개개인의 소득 수준을 완벽하게 설명할 수 있는가?
\pause 불가능함. \pause

\begin{itemize}
\tightlist
\item
  효과(effect): 관측하여 변수로 만들어 모델에 포함한 것들이
  종속변수(소득)의 `체계적'(systematic) 변화에 미치는 영향
\item
  잔차(residuals): 관측한 것들로 설명하지 못하는 불확실성, 변수들의
  효과로 설명하지 못하는 종속변수의 변화
\end{itemize}

\end{frame}

\begin{frame}{Statistical Models and Social Reality}
\protect\hypertarget{statistical-models-and-social-reality-4}{}

통계모델의 잔차는 클 수도, 작을 수도 있음.

\begin{itemize}
\tightlist
\item
  하지만 잔차가 아무리 작다라고 하더라도 모델이 완벽하게 현실을
  설명\(\cdot\)예측할 수 있다고 할 수 없음.
\item
  통계모델은 근본적으로 \emph{서술적(descriptive)} \citep[3]{Fox2016}

  \begin{itemize}
  \tightlist
  \item
    어디까지나 논리적인 기대(이론)와 일치하는 방향으로 경험적인
    관측(데이터)가 배열되어 있는지를 보여줄 뿐임.
  \item
    따라서 통계모델은 항상 불확실성을 내재하지만 그럼에도 변수들의
    체계적 효과, 유의미한 측면을 보여줌.
  \end{itemize}
\end{itemize}

\end{frame}

\begin{frame}{Statistical Models and Social Reality}
\protect\hypertarget{statistical-models-and-social-reality-5}{}

정리하자면, 통계모델은 그 자체로 현실을 직접적으로 보여주는 것이 아니라
간단하게 유의미한 특성들만을 구조화하여 보여줄 뿐

\begin{itemize}
\tightlist
\item
  구조적 조응성을 가지지만 기능적 조응성을 가지지는 않음.
\item
  예) 모형 비행기를 통해 실제 비행기의 구조를 이해하는 데 도움을 얻을 수
  있지만, 그렇다고 모형 비행기가 실제로 날 수 있는 것은 아님.
\end{itemize}

통계모델의 실용성(practice)과 정확성(accuracy)

\begin{itemize}
\tightlist
\item
  데이터(현실)을 정확하게 보여주지 못하는 모델은 쓸모없지만
\item
  정확하게 현실을 보여주는 모델이라고 하더라도 모두 쓸모 있는 것은 아님.

  \begin{itemize}
  \tightlist
  \item
    통계모델 자체는 굉장히 정교하고 정확하더라도 실제 현실을 설명하는 데
    아무 도움이 되지 않을 수 있음.
  \item
    모델 이전에 이론 수립(theory-building)이 중요한 이유
  \end{itemize}
\end{itemize}

\end{frame}

\hypertarget{observation-and-experiment}{%
\section{Observation and Experiment}\label{observation-and-experiment}}

\begin{frame}{Observation and Experiment}
\protect\hypertarget{observation-and-experiment-1}{}

통계모델의 구분

\begin{itemize}
\tightlist
\item
  실험(experimental) 자료

  \begin{itemize}
  \tightlist
  \item
    연구자가 설명변수들을 직접적으로 통제 (randoly assigned)
  \item
    무작위(randomized) 실험을 통한 인과추론
  \item
    처치(treatment)를 제외한 나머지 요인들을 모두 무작위로 통제
  \end{itemize}
\item
  관측(observational) 자료

  \begin{itemize}
  \tightlist
  \item
    설명변수들의 값은 연구자에 의해 관측된 결과
  \item
    설명변수들의 값에 따른 종속변수의 '평균'의 변화 (differences in
    mean)
  \item
    따라서 인과추론은 정당화되지 않음
  \end{itemize}
\end{itemize}

\end{frame}

\begin{frame}{Observation and Experiment}
\protect\hypertarget{observation-and-experiment-2}{}

관측 자료를 통한 통계모델이 인과관계(causality)을 담보하지 않는 이유

\begin{itemize}
\item
  적절한 변수를 모델에서 누락했을 경우

  \begin{itemize}
  \tightlist
  \item
    관측되지 않은 혼재변수(confounding variables)가 존재
  \item
    만약 종속변수에 영향을 미치는 변수인데 모델에 들어가지 않았다면
    통제가 충분하지 못했다는 것
  \item
    만약 누락된 변수가 다른 설명변수들에 영향을 미치는 변수라면 간접적인
    방식으로 종속변수에 영향을 줄 수 있으므로 마찬가지로 통제가 충분히
    이루어지지 않았다는 것
  \end{itemize}
\item
  적절하지 않은 변수를 모델에 포함했을 경우
\item
  통제가 제대로 이루어졌다는 확신을 할 수 없기 때문에 관측자료를
  바탕으로 한 통계분석을 인과적으로 해석하는 데 한계가 존재

  \begin{itemize}
  \tightlist
  \item
    상대적으로 실험자료는 관측자료에 비해 통제가 용이
  \item
    그러나 설명변수의 식별이라는 문제에 있어서는 마찬가지로 불확실성이
    내재
  \item
    어떠한 방법으로도 \emph{모든} 적절한 변수들이 통제되었다고 확신할
    수는 없음.
  \item
    연구 대상인 사회현상과 인간 행태가 가지는 본연적 불확실성
  \end{itemize}
\end{itemize}

\end{frame}

\begin{frame}{Observation and Experiment}
\protect\hypertarget{observation-and-experiment-3}{}

\begin{figure}
\centering
\begin{tikzpicture}
    
    \draw[-{Triangle[scale=1]}, thick] (0,0)--(6,0);
    \node[left] at (0,0.1){Education};
    \draw[-{Triangle[scale=1]}, thick] (0,0.2)--(3.1,5);
    \node[above] at (3.1,5){Income};
    \draw[-{Triangle[scale=1]}, thick] (3.2,5)--(6.2,0.1);
    \node[right] at (6.3,0.1){Prestige};
    
\end{tikzpicture}
\caption{교육, 소득, 직업에 관한 인과모델}
\label{fig1}
\end{figure}

\end{frame}

\begin{frame}{Observation and Experiment}
\protect\hypertarget{observation-and-experiment-4}{}

Figure \ref{fig1}의 이해

\begin{itemize}
\item
  소득 수준은 직업 수준에만 영향을 미치지만, 교육 수준은 소득 수준과
  직업 수준 모두에게 영향을 미침.
\item
  직업 수준과 소득 수준의 관계는 어느 정도 허위적

  \begin{itemize}
  \tightlist
  \item
    교육 수준을 통제했을 때, 소득 \(\rightarrow\) 직업의 효과는 감소할
    것: 허위적 요소의 제거
  \end{itemize}
\item
  반대로 교육 수준과 직업 수준의 관계는 소득 수준에 의해 어느 정도
  매개됨(mediated).

  \begin{itemize}
  \tightlist
  \item
    소득 수준을 통제했을 때, 교육 \(\rightarrow\) 직업의 효과가
    감소하였다면, 그 감소한 만큼이 교육이 소득을 경유해 직업 수준에
    미치는 `간접적 효과'
  \end{itemize}
\end{itemize}

관측자료를 이용할 경우, 종속변수에 가장 우선적인 원인이 되는 설명변수와
그 설명변수-종속변수 관계에 끼어드는 변수를 구별하는 것이 중요

사회과학에서 ``원인''(cause)이라는 개념은 조금 느슨하게 사용됨.

\end{frame}

\hypertarget{populations-and-samples}{%
\section{Populations and Samples}\label{populations-and-samples}}

\begin{frame}{Populations and Samples}
\protect\hypertarget{populations-and-samples-1}{}

\textbf{연구 문제}: 고등학교 3학년 수험생이 공부를 오래할수록 모의고사
성적을 더 잘 받을까?

\begin{itemize}
\item
  모집단: 전체 고등학교 3학년 수험생
\item
  표본: 모의고사 시험을 친 고등학교 3학년 수험생 집단
\end{itemize}

이때, 표본은 전체 고등학교 3학년 수험생에 대한 대표성을 지닌, 그보다
작은 규모의 집단을 의미

\begin{itemize}
\item
  우리가 관심을 가지고 있는 것은 `일반화할 수 있는' 모집단
\item
  하지만 현실에서 모집단을 관측하기란 불가능
\item
  따라서 우리는 모집단을 대표할 수 있는 표본의 특성을 통해 모집단 또한
  그러할 것이라는 추론(inference)을 하게 됨.

  \begin{itemize}
  \tightlist
  \item
    기술추론(descriptive inference): 표본의 생김새를 보고 모집단도
    어떻게 생겼을 것이라는 추론
  \item
    인과추론(causal inference): 표본에서 어떠한 변수들 간의 관계가
    모집단 수준에서도 그러할 것이라는 추론
  \end{itemize}
\end{itemize}

\end{frame}

\begin{frame}{Populations and Samples}
\protect\hypertarget{populations-and-samples-2}{}

통계 추론은 어느 정도 불확실성을 내재한 채로 수집된 자료들이 나타내는
경향성(pattern)의 안정성에 관한 것

\begin{itemize}
\item
  실험 자료를 통해서 우리는 그 관계가 안정적일 때, 인과성(causation)을
  얘기할 수 있음.
\item
  마찬가지로 불확실성을 내재한 자료들을 통해 모집단에 관한
  일반화(generalization)를 할 수 있지만, 어디까지나 가치판단의 문제와
  직결
\end{itemize}

\end{frame}

\begin{frame}[allowframebreaks]{}
\bibliography{Rstat}
\end{frame}


\section[]{}
\frame{\small \frametitle{Table of Contents}
\tableofcontents}
\end{document}
